% Esta es la Plantilla UNAL en LaTeX
\documentclass[10pt,spanish,fleqn,openany,twoside,letterpaper]{book}

%Muestra los márgenes del documento para evitar Warnings
%Para activar la siguiente línea quite el simbolo % 
%\usepackage[showframe]{geometry}

%Formato de fuentes bibliográficas
%Use el estilo bibliográfico que sea pertinente según el área de estudio APA, IEEE, etc

%Usando el paquete BibLaTeX
%Cita normal con \cite[página]{} y cita con paréntesis \parencite[página]{}

% Configuración de BibLaTeX
%\usepackage[backend=biber,style=authoryear,maxcitenames=2,maxbibnames=99,giveninits=true,uniquename=false]{biblatex}
%\addbibresource{biblio.bib}

% Cambiar el idioma de las referencias bibliográficas a español
%\DefineBibliographyStrings{spanish}{%
%  andothers = {et\addabbrvspace al\adddot},
%  andmore = {et\addabbrvspace al\adddot},
%}

% Personalizar el formato de las citas y la bibliografía
%\DeclareNameAlias{sortname}{family-given}
%\DeclareDelimFormat{multinamedelim}{\addcomma\space}
%\DeclareDelimFormat{finalnamedelim}{\addcomma\space\&\space}
%\DeclareFieldFormat{titlecase}{\MakeSentenceCase*{#1}}
%\DeclareFieldFormat[article,inbook,incollection,inproceedings,patent,thesis,unpublished]{title}{\titlecase{#1}}
%\DeclareFieldFormat{journaltitlecase}{\titlecase{#1}}
%\DeclareFieldFormat{pages}{#1}
%\DeclareFieldFormat{volume}{\mkbibbold{#1}}
%\renewbibmacro{in:}{}
%\AtEveryBibitem{\clearfield{month}}

%Usando el paquete Natbib
%Cita normal \cite[página]{} y cita con paréntesis \citep[página]{}
\usepackage{natbib}
\bibpunct{[}{]}{;}{\&}{.}{}
\bibliographystyle{dtvstyle}

%Idioma del documento
%Use main para el idioma principal del documento
\usepackage[main=spanish,english]{babel}

% Carácteres especiales
\usepackage{fontenc}
\usepackage{lmodern}

% Evita ligadura li & fl
\usepackage{microtype}
\DisableLigatures{encoding = *, family = *}

% Otros paquetes de tablas y colores avanzados
\usepackage{amsmath,graphicx,rotating,float,multirow}
\usepackage{longtable}
\setlength{\LTcapwidth}{6in}
\usepackage[utf8]{inputenc}
\usepackage{epsfig,epic,eepic,threeparttable,amscd,here,lscape,tabularx,subfigure}
\usepackage{tabu,array}
\usepackage[rgb]{xcolor}

% Permite ver y configurar los parámetros de la página
\usepackage{layout}
%Hyperref permite ver las secciones del texto
\usepackage[hidelinks]{hyperref}

%Permite incluir código de cualquier lenguaje dentro del texto del documento
\usepackage{minted}
\usepackage{fancyvrb}
\newenvironment{myverbatim}{\Verbatim}{\endVerbatim}

%Genera los comandos de la página de autoría
\newcommand{\studentname}{}
\newcommand{\submissiondate}{}
\newcommand{\academictitle}{}
\newcommand{\resgroupone}{}
\newcommand{\resgrouptwo}{}
\newcommand{\researchtopic}{}
\newcommand{\thesisname}{}
\newcommand{\thesisnameeng}{}
\newcommand{\thesisnamelang}{} %Usar solo si se requiere
\newcommand{\director}{}
\newcommand{\directortitle}{}
\newcommand{\codirector}{} %Usar solo si se requiere
\newcommand{\codirectortitle}{} %Usar solo si se requiere
\newcommand{\issuedate}{}
\newcommand{\palabrasclave}{}
\newcommand{\keywords}{}
\newcommand{\schlusselworter}{}
\newcommand{\palavraschave}{}
\newcommand{\sede}{}
\newcommand{\department}{}
\newcommand{\departmenttwo}{} %Usar solo si se requiere
\newcommand{\faculty}{}
\newcommand{\university}{Universidad Nacional de Colombia}

%Información de la tesis
%Diligenciar aquí los datos para su carga automática donde se requiera en el documento
%En el caso de tesis o trabajos finales, verificar que el título coincida con el aprobado por la Facultad
\renewcommand{\studentname}{Wilson Andrés Pinillo Ipia}
\renewcommand{\thesisname}{Optimización Dinámica de Precios mediante Aprendizaje Reforzado
Profundo}
\renewcommand{\thesisnameeng}{Dynamic Price Optimization using Deep Reinforcement Learning}
%\renewcommand{\thesisnamelang}{Nombre del trabajo o tesis en un tercer idioma} %Usar solo si se requiere
\renewcommand{\issuedate}{2025}
\renewcommand{\submissiondate}{Fecha entrega}
\renewcommand{\director}{Ph.D. John Willian Branch Bedoya}
\renewcommand{\directortitle}{Profesor Titular}
%\renewcommand{\codirector}{Prof. Dr. Co director}
%\renewcommand{\codirectortitle}{Indicar si es Profesor Titular/Asociado}
\renewcommand{\academictitle}{Magíster en Ingeniería - Analítica}
\renewcommand{\resgroupone}{Grupo A (Sigla Grupo Investigación 01) }
\renewcommand{\resgrouptwo}{Grupo B (Sigla Grupo Investigación 02) }
\renewcommand{\researchtopic}{Línea}
\renewcommand{\sede}{Sede Medellín} 
\renewcommand{\department}{Departamento de Ciencias de la Computación y de la Decisión}
\renewcommand{\departmenttwo}{Departamento 2} %Usar solo si es necesario
\renewcommand{\faculty}{Facultad de Minas}

%Palabras clave del documento - Tener presente los Theasurus https://www.thesaurus.com/
%Disponible en 3 idiomas aunque se puede extender a francés o otro idioma
\renewcommand{\palabrasclave}{Use palabras clave que estén en Theasaurus} 
\renewcommand{\keywords}{Use keywords available in Theasaurus}
%\renewcommand{\schlusselworter}{}
%\renewcommand{\palavraschave}{}

% Estilo de los encabezados y pies de página
\usepackage{fancyhdr}
\fancyhf{}%
\pagestyle{fancyplain}
\textheight22.5cm \topmargin0cm \textwidth16.5cm \headheight22pt
\oddsidemargin0.5cm \evensidemargin-0.5cm%
\fancypagestyle{plain}{
\fancyhead[RO,LE]{}
\fancyhead[RE,LO]{\small \textbf{\thesisname}}
\fancyfoot[CO,CE]{\thepage}
}
\pagestyle{fancy}
\fancyhf{}%
\renewcommand{\chaptermark}[1]{\markboth{\thechapter.\; #1}{}}
\renewcommand{\sectionmark}[1]{\markright{\thesection.\; #1}{}}
\fancyhead[LO,RE]{\leftmark}
\fancyhead[RO,LE]{\rightmark}
\fancyfoot[CO,CE]{\thepage}
\thispagestyle{fancy}%

\usepackage{titlesec}
% Permite personalizar los títulos de sección y de capítulos
% hang lo deja en el mismo renglón, display lo despliega
% Elimina el "Capitulo" y deja solo el número
\titleformat{\chapter}[hang]
  {\sffamily\Huge\bfseries}{\thechapter}{0.5cm}{\sffamily\Huge}
\titleformat{\section}[hang]{\sffamily\LARGE}{\thesection}{0.5cm}{}
\titleformat{\subsection}[hang]{\sffamily\Large}{\thesubsection}{0.5cm}{}
\titleformat{\subsubsection}[hang]{\sffamily\large}{\thesubsubsection}{0.5cm}{}
\titleformat{\paragraph}[runin]{\sffamily\normalsize}{}{}{\emph}

%Coloca anexo o apéndice en la Tabla de contenido
\usepackage[toc,page]{appendix}

% Configuración de las páginas en twoside-mode
% Permite ver y configurar los parámetros de la página
\setlength{\voffset}{-0.25in}
\setlength{\headwidth}{467pt}
\setlength{\headheight}{22pt}
\setlength{\oddsidemargin}{0pt}
\setlength{\evensidemargin}{0pt}
\setlength{\marginparwidth}{0pt}
\setlength{\marginparsep}{0pt}
\setlength{\parskip}{2em}
\setlength{\footskip}{20pt}
\setlength{\textheight}{650pt}
\setlength{\textwidth}{467pt}
\setlength{\headsep}{5pt}
\setlength{\parindent}{0pt}
\setlength{\baselineskip}{10pt plus 5pt minus 5pt}
\renewcommand{\theequation}{\thechapter-\arabic{equation}}
\renewcommand{\thefigure}{\textbf{\thechapter-\arabic{figure}}}
\renewcommand{\thetable}{\textbf{\thechapter-\arabic{table}}}

%Ajusta el espacio entre la etiqueta de figuras y tablas y su título en la lista de figuras y en la de tablas 
\usepackage{titletoc} 
\titlecontents{figure}[0em]{}{\thecontentslabel\hspace{1em}}{}{\titlerule*[1pc]{.}\contentspage}
\titlecontents{table}[0em]{}{\thecontentslabel\hspace{1em}}{}{\titlerule*[1pc]{.}\contentspage}

%Define la distancia de la primera linea de un parrafo a la margen
\parindent0cm 

%Espacio entre lineas
\renewcommand{\baselinestretch}{1}

%Permite personalizar el ajuste vertical mediante cajas
\usepackage{adjustbox}

%Para rotar texto, objetos, tablas y páginas.
\usepackage{rotating}

%Permite incluir mecanismos y reacciones químicas
\usepackage{tikz}
\usepackage{chemformula}
\usepackage{chemfig}

\usetikzlibrary{calc,arrows.meta}% per right to e left to
\tikzset{
myedge/.style={->, -{Latex[#1]}}
}

%Fuente de la presentación Ancizar Sans UNAL
%Para usar este compilado en Overleaf se debe usar el compilador XeLaTeX o LuaLaTeX!!
%Menu -> Compiler -> XeLaTeX o LuaLaTeX
%La siguiente línea debe comentarse si desea compilar con pdfLaTeX
%\RequireXeTeX

% Definición de la fuente Ancizar Sans
\newif\ifxetexorluatex

\ifxetexorluatex
  \usepackage{fontspec}
  \usefonttheme{serif}
  \setmainfont{AncizarSans}[Path=./AncizarSans/,Scale=1,Extension=.otf,UprightFont=*-Regular,BoldFont=*-Bold,ItalicFont=*-Italic,BoldItalicFont=*-BoldItalic]
\else
  % Si se compila con pdfLaTeX, cargar la fuente apropiada aquí
  \usepackage[T1]{fontenc}
\fi
% Metadatos del documento
\AtBeginDocument{%
	\hypersetup{
		pdfborder={0 0 0},
		pdfauthor={\studentname},
		pdfsubject={\thesisname}, 
		pdfcreator={\studentname},
		pdfproducer={\studentname},
	}
}

%Carga el simbolo de grado y el de Angstrom
\newcommand{\angstrom}{\textup{\AA}}
\newcommand{\grad}{$^{\circ}$}

%Inicio del documento, no olvide la etiqueta de cierre al final \end{document}
\begin{document}

%Nombres y formatos de títulos, tablas y figuras
%Use \sffamily para dejar con letra Sans Serif, sin etiqueta queda LaTeX clásico
\renewcommand{\listfigurename}{\sffamily Lista de figuras}
\renewcommand{\listtablename}{\sffamily Lista de tablas}
\renewcommand{\contentsname}{\sffamily Contenido}
\renewcommand{\chaptername}{\sffamily Capítulo}
\renewcommand{\tablename}{\scriptsize \centering \textbf{Tabla}}
\renewcommand{\figurename}{\scriptsize \centering \textbf{Figura}}
\renewcommand{\appendixname}{\sffamily Anexo}

%Cambia el nombre de la sección de referencias
\renewcommand{\bibname}{\sffamily Referencias Bibliográficas}

%Páginas de Presentación del documento - No modificar esto se hace automáticamente
{\newpage
\thispagestyle{empty}
\begin{center}
\begin{figure}
\centering
\epsfig{file=00Figuras/00f00EscudoUN2016,scale=1}%
\end{figure}
\vspace{2.5cm}
\textbf{\Huge \thesisname} \\ 
\vspace{2.5cm}
\textbf{\Large \studentname} \\
\vspace{5.0cm}
\faculty \\ \department \\
\sede, Colombia \\
\issuedate
\newpage 
\thispagestyle{empty}
\vspace{2.0cm}
%\small Nota: No utilizar el tipo de letra Ancizar en el documento puesto que este tipo de fuente restringe la copia de los archivos en el Repositorio Institucional. [Recuerde borrar esta nota]\\
\textbf{\Huge \thesisname} \\
\vspace{2.0cm}
\textbf{\Large \studentname} \\
\vspace{2.0cm}
\small Trabajo final presentado como requisito parcial para optar por el título de: \\
{\bfseries \academictitle}\\
\vspace{2.0cm}
\textbf{Director:} \\
\director \\
\directortitle \; - \departmenttwo \\
\faculty \\
\university \\ 
\vspace{0.5cm}
\textbf{Codirector(a):} \\
\codirector \\
\codirectortitle \; - \department \\
\faculty \\
\university 
\vspace{1.5cm} \\
\textbf{Línea de investigación:} \\ 
\researchtopic\\
\textbf{Grupo de investigación:} \\
\resgroupone \\
\resgrouptwo \\
\vspace{1.5cm} 
\university \\
\faculty \\
\department \\
\issuedate
\end{center}

% Dedicatorias
\newpage
\thispagestyle{empty}
\begin{flushright}
\begin{minipage}{12.5cm}
\noindent
\\[10em]
%Modificar la cita que se quiere agregar
{\Large Cita 01.}
\\[3em]
Autor
\\ \textit{Fuente}
\\[10em]
%Para anular la adición de una segunda cita anule las siguientes lineas desde acá mediante comentario (%)
{\Large \textit{Wenn du es nicht einfach erkl\"{a}ren kannst, hast du es nicht genug verstanden} - Si no eres capaz de explicar algo claramente, es que aún no lo has entendido lo suficiente.}
\\[3em]
Albert Einstein
%Hasta acá!
\end{minipage}
\end{flushright} 

% Declaracíon de originalidad del texto y del contenido
% No modificar, se hace automáticamente con los comandos ya definidos
\newpage
\chapter*{\sffamily Declaración}
\par Me permito afirmar que he realizado ésta tesis de manera autónoma y con la única ayuda de los medios permitidos y no diferentes a los mencionados el presente texto. Todos los pasajes que se han tomado de manera textual o figurativa de textos publicados y no publicados, los he reconocido en el presente trabajo. Ninguna parte del presente trabajo se ha empleado en ningún otro tipo de tesis. 
\\[1em]
\sede., \submissiondate
\\[6em]
\rule{6cm}{0.5pt}\\
\studentname
}

%Páginas preámbulo, listado de figuras, tablas y tabla de contenido
{\pagestyle{plain} \pagenumbering{roman}
\setlength{\parskip}{1mm}
\chapter*{\sffamily Agradecimientos}
\addcontentsline{toc}{chapter}{Agradecimientos}%
% Comentar las dos lineas de abajo con % en caso que no se requieran abreviaturas y resumen en el trabajo
\chapter*{\sffamily Listado de símbolos y abreviaturas}
\addcontentsline{toc}{chapter}{Listado de símbolos y abreviaturas}
\include{00ResumenAbstract}
% Dejar esta parte así para que genere correctamente la página de la tabla de contenido
\addcontentsline{toc}{chapter}{Lista de figuras}
\listoffigures
\clearpage
\addcontentsline{toc}{chapter}{Lista de tablas}
\listoftables
\clearpage
\addcontentsline{toc}{chapter}{Contenido}
\tableofcontents
\clearpage
}

{\pagenumbering{arabic}
\setlength{\parskip}{\baselineskip}
%Incluir secciones del documento de aqui en adelante
%Use \include para incluir desde una página nueva e \input para incluir sin salto de página
\chapter{Cómo utilizar esta plantilla?}
\par Para escribir en \LaTeX \; se hace exactamente lo mismo que en cualquier procesador de texto, como Microsoft Word, solo se necesita escribir. La diferencia es, que las funcionalidades adicionales que puedan requerirse, se invocan en el código mediante unas extensiones del programa denominadas \textbf{paquetes}.
\par Estas extensiones se invocan en la sección inicial del documento, en nuestro caso en el archivo llamado \textbf{0000.tex}. En este archivo ya están precargadas muchas funcionalidades que dan formato y facilitan el uso de esta plantilla. Sin embargo, es posible adicionar más funciones usando el comando:
\begin{minted}{latex}
\usepackage[Atributos del paquete]{Nombre del paquete}
\end{minted}
\par Cada paquete disponible para \LaTeX \, dispone de información detallada de libre acceso, que puede se consultada en CTAN, buscando en internet por el nombre del paquete.
\section{Algunos comandos útiles}
\par Durante la escritura se pueden requerir algunas cosas sencillas que facilitan la tarea y que pueden constituir trucos de uso:
\subsection{Insertar comentarios}
\par Insertar comentarios en \LaTeX es muy sencillo, para comentar código se usa el signo de porcentaje \textbf{\%}.
\begin{minted}{latex}
% Esta linea no se compila en el documento, está comentada
\end{minted}
\par Los comentarios pueden servir para hacer anotaciones, anular segmentos de código que no se desean borrar del documento; pero que tampoco deben aparecer en el texto terminado, etc.
\pagebreak
\subsection{Dar formato a los capítulos y secciones}
\par Para dar formato a los capítulos se sigue esta estructura:
\begin{minted}{latex}
\chapter{Nombre del Capítulo} 
% Capitulo, división de primer orden en el texto
\chapter*{Nombre} 
% Capitulo requerido en el texto pero que no debe aparecer en la Tabla de Contenido
\section{Nombre} 
% Sección del capitulo, división de segundo orden
\section*{Nombre}
% Sección requerida en el texto pero que no debe aparecer en la Tabla de Contenido
\subsection{Nombre} 
% Subsección de una sección de un capitulo, división de tercer orden
\subsubsection{Nombre} 
% Subsección de una subsección, división de cuarto orden
\paragraph{Nombre} 
% Puede tomarse como división de quinto orden o para resaltar información importante
\end{minted}
\par La ventaja de \LaTeX respecto a otros procesadores de texto como Word, es que el proceso de numeración se realiza de \textbf{forma automática} mientras se compila el documento, por lo tanto, el usuario no debe preocuparse de que se altere de forma impredecible la numeración asignada. Esto aplica a capítulos, figuras, tablas, ecuaciones, etc.
\par Para escribir los párrafos se escribe normalmente el texto, pero se puede generar el espaciado automático usando el comando \mintinline{latex}{\par}
\begin{minted}{latex}
\par 
% Se debe colocar antes de la primera palabra del parrafo y separada con un espacio sencillo
\end{minted}
\par Este comando ya genera el espaciado entre los párrafos, acorde con lo establecido en el documento \textbf{0000.tex}
\section{Entornos comunes}
\par Para la inserción de tablas, figuras, gráficas, listados, entre otras, \LaTeX usa un conjunto de códigos que se denominan \textbf{entornos}. Estos entornos tienen como característica básica que poseen una etiqueta de inicio y una etiqueta de cierre que suelen llevar la sintaxis: \mintinline{latex}{\begin{} ... \end{}}.
\par Mientras que las imágenes y las tablas pueden ir dentro o fuera de entornos según se requiera, otros elementos como listados, cambios de orientación o de alineación, necesariamente deben tener las etiquetas de apertura o de cierre.
\subsection{Insertar imágenes}
\par Para insertar imágenes en el texto se tienen dos alternativas, la primera es la inserción de solo la imagen, esto se hace con el comando \mintinline{latex}{\includegraphics[]{}}.
\begin{minted}{latex}
% Opción 1 - Inserción simple de una imagen 
\includegraphics[Parámetros de la imagen]{Fuente o lugar de la imagen}
% Algunos parámetros útiles que se usan en el recuadro [Parámetros] son:
% scale=Valor entre 0 y 1, la imagen se proporciona basada en el archivo original
% \textwidth, la imagen queda con un ancho máximo igual al que tiene el texto
% Esta imagen no va a quedar registrada en el listado de figuras de la plantilla
\end{minted}
\par Para insertar una Figura que si deba aparecer en el listado de figuras de esta plantilla se debe invocar el entorno \textbf{figure}. (\mintinline{latex}{\begin{figure} ... \end{figure}})
\begin{minted}{latex}
% Opción 2 - Inserción de una Figura que SI debe estar en la Tabla de Contenido
\begin{figure}[ht]
    % Usar [ht] (here, top) esto coloca la tabla en la ubicación donde se solicita
    % O en su defecto en la parte superior de la siguiente página (top)
    \centering
    \includegraphics{} % Mismo comando del caso anterior
    \caption[Titulo de la figura abreviado]{Titulo extendido de la figura}
    % El nombre abreviado de la tabla se ve en la Lista de Figuras
    % El nombre extendido de la tabla se ve en el texto del documento
    \label{fig:Etiqueta} 
    % Etiqueta de la figura, permite citarla en el texto usando \ref{fig:Etiqueta}
    % La etiqueta es definida por el usuario y DEBE ser única
\end{figure}
\end{minted}
\par Se sugiere usar la carpeta \textbf{00Figuras} para almacenar las imágenes
\subsection{Insertar tablas}
\par Al igual que para las imágenes, las tablas se pueden insertar solas usando el entorno \textbf{tabular}, o también insertarlas de forma que se registren en el Listado de Tablas del documento, en ese caso se usa el entorno \textbf{table} (\mintinline{latex}{\begin{table} ... \end{table}}), que a su vez contiene al entorno \mintinline{latex}{\begin{tabular} ... \end{tabular}}
\begin{minted}{latex}
% Opción 1: Insertar una tabla sencilla con tabular
\begin{tabular}{c|c} %Tabla de 2 x 2 con texto centrado
   Celda 1-1 & Celda 1-2  \\
   Celda 2-1 & Celda 2-2
\end{tabular}
% Opción 2: Insertar una Tabla que quede en la Lista de Tablas del documento
\begin{table}[ht] 
% Usar [ht] (here, top) esto coloca la tabla en la ubicación donde se solicita
% O en su defecto en la parte superior de la siguiente página (top)
    \centering % Formato centrado 
    \begin{tabular}{l|r} 
    % Tabla 2 x 2 Texto a la Izquierda Columna 1, Derecha Columna 2 
         &  \\
         & 
    \end{tabular}
    \caption[Nombre abreviado de la Tabla]{Nombre extendido de la tabla}
    % El nombre abreviado de la tabla se ve en la Lista de Tablas
    % El nombre extendido de la tabla se ve en el texto del documento
    \label{tab:Etiqueta}
    % Etiqueta de la tabla, permite citarla en el texto usando \ref{tab:Etiqueta}
    % La etiqueta es definida por el usuario y DEBE ser única
\end{table}
\end{minted}
\par Se sugiere usar la carpeta \textbf{00Tablas} para almacenar las tablas, en especial aquellas que sean muy grandes o tengan códigos complicados.
\par Para generar las tablas se puede usar herramientas como \href{https://www.tablesgenerator.com/}{Table Generator}, que permiten copiar directamente desde Word o Excel.
\subsection{Insertar listados}
\par Una opción muy útil en cualquier texto es insertar listados, esto se hace con dos entornos. \textbf{enumerate} para listas numeradas e \textbf{itemize} para listados de categorías (sin numerar)
\par Una lista sin números usando el entorno \mintinline{latex}{\begin{itemize} ... \end{itemize}} 
\begin{myverbatim}
\begin{itemize}
    \item Primer elemento de la lista
    \item Segundo elemento de la lista
\end{itemize}
\end{myverbatim}
\par Una lista sin números usando el entorno \mintinline{latex}{\begin{enumerate} ... \end{enumerate}} 
\begin{myverbatim}
\begin{enumerate}
    \item Primer elemento de la lista
    \item Segundo elemento de la lista
    \item Agregar tantos elementos como se requiera
\end{enumerate}
\end{myverbatim}
\section{Información Adicional}
\par Si buscas conocer más información de como esta estructurado \LaTeX se recomienda ver los siguientes recursos:
\begin{itemize}
    \item \cite{DeCastroKorgi2010} - El Universo \LaTeX - Facultad de Ciencias - Universidad Nacional de Colombia
    \item \href{https://users.monash.edu.au/~anam/webcurso/curso_archivos/LaTeX/2EstructuraEdicion.pdf}{Estructura y Texto en \LaTeX}
    \item \href{https://www.overleaf.com/learn/how-to/Creating_a_document_in_Overleaf}{Cómo crear un documento en Overleaf}
\end{itemize} % Anular esta linea con comentario de ser necesario
\chapter{Planteamiento del Problema}
\clearpage
\newpage
\chapter{Hipótesis}
\include{00Objetivos}
\include{01Seccion01}
\chapter{Marco teórico}%Estado del arte
\include{03Seccion03}
\include{04Seccion04}
\chapter{Discusión de resultados}
\include{06Seccion06}
\include{07Seccion07}

%Inicio del apéndice o anexos
\begin{appendix}
\include{08Apendice01}%
\end{appendix}

%Permite visualizar la bibliografía en la tabla de contenido
%Cambie el nombre a Bibliografía o Literatura Citada en la siguiente línea de ser preciso
\addcontentsline{toc}{chapter}{Referencias Bibliográficas} 

\let\OLDthebibliography=\thebibliography
\def\thebibliography#1{\OLDthebibliography{#1}}
{\scriptsize
\pagestyle{plain}
% Nombre del documento donde se almacenan las referencias
\bibliography{Referencias}
\nocite{*}
% Inserta un página adicional al final en blanco 
%\cleardoublepage
% Para NO insertar una página adicional al final usar \clearpage
\clearpage
}}

\end{document}